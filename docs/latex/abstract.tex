\chapter*{\centering{\emph{\large{ABSTRACT}}}}
\singlespacing{}

\textbf{MUHAMMAD HAFIZ HISBULLAH}, Detection of Chronic Wound Perimeter Using the 
GrabCut Algorithm. Undergraduate Thesis, Computer Science Program, Faculty of 
Mathematics and Natural Sciences, State University of Jakarta. January 2024.
\\

Chronic wounds pose a complex health issue, particularly for patients with conditions 
such as Diabetes Mellitus (DM). The wound healing process involves meticulous assessment 
and effective management, yet manual methods in wound measurement often prove inaccurate 
and time-consuming. This thesis aims to explore the application of the GrabCut method 
in image processing to detect the perimeter of chronic wounds. This method holds 
the potential to provide objective and reliable wound assessment analysis. Leveraging 
image processing technology and machine learning, we attempt to overcome the limitations 
of manual methods by employing the GrabCut algorithm. The initial step involves 
capturing wound photos using mobile devices, followed by color calibration to ensure 
consistency. The GrabCut method is then applied to segment the wound area, focusing 
on the separation between the main object (the wound) and the background. The final results 
indicate that data successfully segmented using GrabCut is 70 out of 71, 
surpassing the previous study conducted by \cite{Rizki:2022} using the snake method, 
both integer version (12 out of 71) and interpolation version (44 out of 71). 
Additionally, GrabCut yields an average accuracy of 89.36\%, while the snake method 
achieves 77.18\% for the integer version and 86.1\% for the interpolation version.
\\
\\
\textbf{Keywords}: chronic wounds, GrabCut, image processing, wound assessment, wound management.
