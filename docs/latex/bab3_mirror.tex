\section{Algoritma Segmentasi Gambar dengan Metode \emph{GrabCut}}

Algoritma \emph{GrabCut} terdiri dari gabungan beberapa algoritma lain, yaitu 
pemrosesan oleh algoritma GMM dan minimisasi energi oleh algoritma \emph{GraphCut},
berikut algoritma dari \emph{GrabCut}:  

\begin{algorithm}                     
\caption{Algoritma segmentasi gambar dengan \emph{GrabCut} (\cite{Rother:2004})}          
\label{algo:grabcut}                          
\begin{algorithmic}                    % enter the algorithmic environment
    % \State $\textbf{From \xspace} GaussianMixtureModels \textbf{\xspace Import \xspace} GaussianMixture$
    \State $gambar \gets \Call{loadImage}$  \Comment{Input gambar 2.jpg}
    \State $\textbf{bool } kotak \gets false$ \Comment{Inisiasi keberadaan kotak}
    \State $\textbf{bool } drawing \gets false$ \Comment{Inisiasi keberadaan brush}
    \State $iy, ix \gets int$
    \State $gambar2 \gets \Call{Copy}{gambar}$
    \State $mask \gets \Call{np.zeros}{gambar.shape[:2], dtype=np.uint8}$    
    \State $piksel\textunderscore komponen \gets \Call{np.zeros}{gambar.shape[0], gambar.shape[1]}$
    \State $\textbf{array } kotak\textunderscore lok[4] \gets [...]$
    \\
    \State $map<string, string> flagsColor$
    \State $map<string, integer> flagsValue$
    
    \State $flagsColor['bg'], flagsColor['fg']  \gets 'black', 'white'$
    % \State $flagsColor['fg'] \gets 'white'$

    \State $flagsValue['bg'], flagsValue['fg'], flagsValue['prob\textunderscore bg'], flagsValue['prob\textunderscore fg'] \newline
     \gets 0, 1, 2, 3$
    % \State $flagsValue['bg'] \gets 0$
    % \State $flagsValue['prob\textunderscore bg'] \gets 2$
    % \State $flagsValue['prob\textunderscore fg'] \gets 3$

    % \State $flags['bg'] \gets ['black', 0]$
    % \State $flags['fg'] \gets ['white', 1]$
    % \State $flags['pr\textunderscore bg'] \gets ['red', 2]$
    % \State $flags['pr\textunderscore fg'] \gets ['blue', 3]$
    \\
    \Function{MouseHandler}{$event, x, y, flagsColor, flagsValue, param$}
        \If{$event == \Call{cv}{EVENT\textunderscore RBUTTON}$} 
            \State $ix, iy = x, y$
            \State $\Call{cv.rectangle}{gambar, (ix, iy), (x, y), flagsColor['warna'], 2}$
            \State $kotak \gets True$
            \State $kotak\textunderscore lok[4] \gets [min(ix, x), min(iy, y), abs(ix-x), abs(iy-y)]$
        \EndIf 
        \If{$event == \Call{cv}{EVENT\textunderscore LBUTTON}$} 
            \State $drawing \gets True$
            \State $\Call{cv.circle}{gambar, (x, y), 3, flagsColor['warna'], -1}$
            \State $\Call{cv.circle}{mask, (x, y), 3, flagsValue['value'], -1}$
            \State $drawing \gets False$
        \EndIf 
    \EndFunction
    \\
\algstore{algo:algo_all}
\end{algorithmic}
\end{algorithm}

\begin{algorithm}                     
\begin{algorithmic}                    % lanjutan algo_all di atas
\algrestore{algo:algo_all}

    \\
    \Comment{(Algorithm \ref{algo:intialize}) Inisiasi TU = 1 untuk piksel didalam kotak, gambar \ref{gambar:2.6} }
    \If{$kotak\textunderscore lok[4] \neq None $} 
        \State $mask[kotak\textunderscore lok[...]] = 1$ 
    \EndIf

    \Function{inisasi\textunderscore piksel}{$mask$}
        \State $...$
    \EndFunction
    \\ 
    \\
    \Comment{(Algorithm \ref{algo:assign_GMM})Inisiasi dan \emph{Assign} GMM ke setiap piksel, tahap 1 gambar \ref{gambar:2.6} }
    \Function{init\textunderscore assign\textunderscore gmm}{$idx\textunderscore bg, idx\textunderscore fg$}
        \State $gmm\textunderscore bg, gmm\textunderscore fg\gets GaussianMixture(...)$
        \State $idx\textunderscore bg \gets ...$
        \State $idx\textunderscore fg \gets ...$
    \EndFunction

    % \Function{init\textunderscore assign\textunderscore gmm}{$idxComponents,gmm\textunderscore bg, gmm\textunderscore fg$}
    %     \State $idx\textunderscore bg \gets ...$
    %     \State $idx\textunderscore fg \gets ...$
    % \EndFunction
\end{algorithmic}
\end{algorithm}

\begin{algorithm}                     
    \caption{Algoritma inisialisasi \emph{bounding box} pada area luka}          
    \label{algo:intialize}                          
    \begin{algorithmic}                    % enter the algorithmic environment
        \\  \Comment{Tahap inisialisasi pada gambar \ref{gambar:2.6}}
    
        \Function{inisasi\textunderscore piksel}{$mask$}

            \If {$kotak\textunderscore lok \neq None$}
                \State $mask[kotak\textunderscore lok[1]:kotak\textunderscore lok[1] + 
                kotak\textunderscore lok[3],\newline  
                \hspace*{4em} kotak\textunderscore lok[0]: kotak\textunderscore lok[0] + 
                kotak\textunderscore lok[2]] \gets flags['fg']['value']$
            \EndIf

            \State $idx\textunderscore bg \gets \textbf{where}(mask == flags['bg']['value']  \textbf{\xspace or\xspace}  \newline 
            \hspace*{4em} mask == flags['pr\textunderscore bg']['value'])$
            \State $idx\textunderscore fg \gets \textbf{where}(mask == flags['fg']['value']  \textbf{\xspace or\xspace}  \newline 
            \hspace*{4em} mask == flags[''pr\textunderscore fg']['value'])$
        
        \EndFunction
    % \algstore{algo:initialize}
    \end{algorithmic}
\end{algorithm}
        
    % \begin{algorithm}                     
    %     \begin{algorithmic}                    % enter the algorithmic environment
    %     \algrestore{algo:initialize}
    %     \\
    %     \Comment{Tahap inisialisasi pada gambar \ref{gambar:2.6}}
    %     \If {$kotak\textunderscore lok \neq None$}
    %         \State $mask[kotak\textunderscore lok[1]:kotak\textunderscore lok[1] + 
    %         kotak\textunderscore lok[3],\newline  
    %         \hspace*{4em} kotak\textunderscore lok[0]: kotak\textunderscore lok[0] + 
    %         kotak\textunderscore lok[2]] \gets flags['fg']['value']$
    %     \EndIf
    
    %     \Function{inisasi\textunderscore piksel}{}
    %         \State $idx\textunderscore bgd \gets \textbf{where}(mask == flags['bg']['value']  \textbf{or} \newline
    %         \hspace*{4em} mask == flags['pr\textunderscore bg']['value'])$
    %         \State $idx\textunderscore fgd \gets \textbf{where}(mask == flags['fg']['value']  \textbf{or} \newline
    %         \hspace*{4em} mask == flags[''pr\textunderscore fg']['value'])$
    %     \EndFunction
    %     \\
    %     \Comment{Tahap 1 pada gambar \ref{gambar:2.6}}
    %     \end{algorithmic}
    % \end{algorithm}

\begin{algorithm}                     
    \caption{Algoritma \emph{Assign} GMM pada setiap piksel citra}          
    \label{algo:assign_GMM}                          
    \begin{algorithmic}                    % enter the algorithmic environment

    % \State $\textbf{int } komponen\textunderscore gmm \gets 5$
    % \State $coefs \gets  \Call{np.zeros}{komponen\textunderscore gmm}$
    % \State $means \gets \Call{np.zeros}{komponen\textunderscore gmm, target.shape[2]}$
    % \State $covarians \gets \Call{np.zeros}{komponen\textunderscore gmm, target.shape[2], target.shape[2]}$
    % \State $\textbf{int } k$

    % \State $\textbf{Eigen::MatrixXd } coefs(komponen\textunderscore gmm)$
    % \State $\textbf{Eigen::MatrixXd } means(komponen\textunderscore gmm, target.shape[2])$

    % \Function{initGMM}{$idx\textunderscore bg, idx\textunderscore fg$}
    %     \State $GMM\textunderscore BG \gets GaussianMixture(gambar[idx\textunderscore bg])$
    %     \State $GMM\textunderscore FG \gets GaussianMixture(gambar[idx\textunderscore fg])$
    % \EndFunction

    \Function{init\textunderscore assign\textunderscore gmm}{$idx\textunderscore bg, idx\textunderscore fg$}
        \State $GMM\textunderscore BG \gets GaussianMixture(gambar[idx\textunderscore bg])$
        \State $GMM\textunderscore FG \gets GaussianMixture(gambar[idx\textunderscore fg])$

        \State $piksel\textunderscore komponen[idx\textunderscore bg] \gets \Call{GMM\textunderscore BG.dis\textunderscore mult}{gambar[idx\textunderscore bg]}$
        \State $piksel\textunderscore komponen[idx\textunderscore fg] \gets \Call{GMM\textunderscore FG.dis\textunderscore mult}{gambar[idx\textunderscore fg]}$
    \EndFunction

    \Function{init\textunderscore rand}{$target$}
        \State $\textbf{vector<int> } labelPix(target.shape[0])$
        \For{$\textbf{int } i = 0; i < target.shape[0]; i++$}
            \State $labelPix[i] \gets rand() \mod 5;$
        \EndFor
    \EndFunction

    \Function{dis\textunderscore mult}{$target$} \Comment{rumus \ref{eq:distribusi_multi}}
        \State $gauss\textunderscore score \gets  \Call{np.zeros}{target.shape[0]}$
        
        \For{$k = 0; k < komponen\textunderscore gmm; k++$ }
            \If{$coefs > 0$}
                \State $XminMu \gets target - means[k]$
                \State $tmp\textunderscore dot \gets \Call{np.dot}{covarians[k].inv, XminMu.T}$
                \State $tmp\textunderscore mult \gets \Call{np.einsum}{'ij,ij->i', XminMu, tmp\textunderscore dot.T}$
                \State $pembagi \gets \Call{np.sqrt}{2* NP.PI} * \Call{np.sqrt}{covarians[k].det}$
                \State $gauss\textunderscore score \gets \frac{\Call{np.exp}{-0.5 * tmp\textunderscore mult}}{pembagi}$
            \EndIf
        \EndFor

        \State $\textbf{return } \Call{np.argmax}{gauss\textunderscore score}$
    \EndFunction
    % \algstore{algo:initialize}
    \end{algorithmic}
\end{algorithm}
