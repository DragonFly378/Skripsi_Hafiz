%!TEX root = ./template-skripsi.tex
%-------------------------------------------------------------------------------
%                            	BAB IV
%               		KESIMPULAN DAN SARAN
%-------------------------------------------------------------------------------

\chapter{HASIL DAN PEMBAHASAN}

\section{Pemrosesan citra gambar input}
Langkah pertama dalam pemrosesan citra gambar input adalah menentukan rasio piksel 
dari citra tersebut, penulis mengubah lebar citra menjadi 320 piksel untuk efisiensi 
serta menyamakan ukuran tiap sampel citra gambar, berikut \emph{source code} nya:

\begin{figure}[H]
	\begin{lstlisting}[language=Python, basicstyle=\tiny]
		def setSizeImg(self):
			global new_width, aspect_ratio, new_height
			# set ukuran gambar
			new_width = 320  # Atur ukuran hanya 320 px
			aspect_ratio = self.image.width / self.image.height
			new_height = int(new_width / aspect_ratio)
		
		# Laod image
		self.image = Image.open(self.image_path)

		# Resize ukuran gambar
		self.setSizeImg() 
		self.image = self.image.resize((new_width, new_height)) 
	\end{lstlisting}
	\caption{\emph{source code} mengubah ukuran lebar citra menjadi 320 piksel}
	\label{img:resize_gambar}
\end{figure}

% Baris ini digunakan untuk membantu dalam melakukan sitasi
% Karena diapit dengan comment, maka baris ini akan diabaikan
% oleh compiler LaTeX.
\begin{comment}
\bibliography{daftar-pustaka}
\end{comment}