%!TEX root = ./template-skripsi.tex
%-------------------------------------------------------------------------------
%                            	BAB IV
%               		KESIMPULAN DAN SARAN
%-------------------------------------------------------------------------------

\chapter{HASIL DAN PEMBAHASAN}

\section{Pemrosesan citra gambar input}
Langkah pertama dalam pemrosesan citra gambar input adalah menentukan rasio piksel 
dari citra tersebut, penulis mengubah lebar citra menjadi 320 piksel untuk efisiensi 
serta menyamakan ukuran tiap sampel citra gambar, berikut \emph{source code} nya:

\begin{figure}[H]
	\begin{lstlisting}[language=Python, basicstyle=\tiny]
		def setSizeImg(self):
			global new_width, aspect_ratio, new_height
			# set ukuran gambar
			new_width = 320  # Atur ukuran hanya 320 px
			aspect_ratio = self.image.width / self.image.height
			new_height = int(new_width / aspect_ratio)
		
		# Laod image
		self.image = Image.open(self.image_path)

		# Resize ukuran gambar
		self.setSizeImg() 
		self.image = self.image.resize((new_width, new_height)) 
	\end{lstlisting}
	\caption{\emph{source code} mengubah ukuran lebar citra menjadi 320 piksel}
	\label{img:resize_gambar}
\end{figure}

Fungsi \texttt{setSizeImg()} dibuat untuk mengubah ukuran lebar piksel citra 
menjadi 320 piksel, untuk mempertahankan rasionya, makan dibentuk dengan menghitung
menggunakan rumus \texttt{aspect\_ratio} dimana tinggi gambar akan otomatis mengikuti lebar gambar 
dengan perbandingan yang stabil.

\section{Deteksi keliling luka menggunakan \emph{Grabcut}}
Alur deteksi keliling luka menggunakan algoritma \emph{Grabcut} dilakukan melalui
beberapa tahapan, setiap tahapan akan mengolah data input berupa citra gambar luka 
yang disimpan menjadi \emph{array} multidimensi (lebar, tinggi, nilai RGB).
Berikut adalah tahapannya:

\subsection{Inisiasi kotak (\emph{Bounding Box})}

Dalam segmentasi citra luka, dimulai dengan inisiasi kotak atau \emph{bounding box} 
yang digambar mengelilingi area citra luka, berikut \emph{source code} nya:

\begin{figure}[H]
	\begin{lstlisting}[language=Python, basicstyle=\tiny]
		def drawing_rectangle(self):
			print("mulai gambar kotak")
			self.main_canvas.bind("<ButtonPress-1>", self.onClick_rect)
			self.main_canvas.bind("<B1-Motion>", self.onDrag_rect)
			self.main_canvas.bind("<ButtonRelease-1>", self.onRelease_rect)

		def onClick_rect(self, event):
			print("Keberadaan kotak: ",KOTAK["is_drawn"])
			if KOTAK["is_drawn"] is not True:
				KOTAK["titik_start"] = (event.x, event.y)
			else:
				print("kotak sudah ada")

		def onDrag_rect(self, event):
			if KOTAK["is_drawn"] is not True:
				KOTAK["titik_akhir"] = (event.x, event.y)
				self.update_image()

		def onRelease_rect(self, event):
			if KOTAK["titik_start"] and KOTAK["titik_akhir"]:
				KOTAK["coord"] = (self.get_rectangle_coords())
				print("Koordinat Kotak: ", KOTAK["coord"])
				KOTAK["titik_start"] = None
				KOTAK["titik_akhir"] = None
				KOTAK["is_drawn"] = True
			print("Keberadaan kotak: ",KOTAK["is_drawn"])
			print("Koordinat Kotak: ", KOTAK["coord"])

		def get_rectangle_coords(self):
			if KOTAK["titik_start"] and KOTAK["titik_akhir"]:
				x1, y1 = KOTAK["titik_start"]
				x2, y2 = KOTAK["titik_akhir"]
				return (x1, y1, x2, y2)
			else:
				return None

		def drawing_rectangle(self):
			print("mulai gambar kotak")
			self.main_canvas.bind("<ButtonPress-1>", self.onClick_rect)
			self.main_canvas.bind("<B1-Motion>", self.onDrag_rect)
			self.main_canvas.bind("<ButtonRelease-1>", self.onRelease_rect)

	\end{lstlisting}
	\caption{\emph{source code} menggambar kotak pada area luka}
	\label{img:inisiasi_rectangle}
\end{figure}

Ketika kotak sudah tergambar, koordinat dimasukkan kedalam variabel \texttt{KOTAK["coord"]}
pada fungsi \texttt{onRelease\_rect()} yang mana \texttt{KOTAK["coord"]} berupa array 
berisi x1, y1, x2, y2 yang akan digunakan untuk tahap selanjutnya.

\subsection{Inisiasi piksel}
Tahap selanjutnya adalah membuat salinan \emph{masking} pada citra gambar luka yang diubah
menjadi gambar hitam, salinan dibuat dengan menggunakan \emph{library} \emph{Numpy}
dengan target citra gambar luka. Berikut \emph{source code} nya:


\begin{figure}[H]
	\begin{lstlisting}[language=Python, basicstyle=\tiny]
			self.gambar = np.array(self.image)
        self.gambar2 = self.gambar.copy()
        print("ukuran gambar: ", self.gambar.shape)

        # buat masking dari gambar awal
        self.mask = np.zeros(self.gambar2.shape[:2], dtype=np.uint8)
        self.mask2 = self.mask.copy()
	\end{lstlisting}
	\caption{\emph{source code} menggambar kotak pada area luka}
	\label{img:mask_image}
\end{figure}




% Baris ini digunakan untuk membantu dalam melakukan sitasi
% Karena diapit dengan comment, maka baris ini akan diabaikan
% oleh compiler LaTeX.
\begin{comment}
\bibliography{daftar-pustaka}
\end{comment}