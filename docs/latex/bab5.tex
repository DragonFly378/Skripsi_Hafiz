%!TEX root = ./template-skripsi.tex
%-------------------------------------------------------------------------------
%                            	BAB IV
%               		KESIMPULAN DAN SARAN
%-------------------------------------------------------------------------------

\chapter{KESIMPULAN DAN SARAN}

\section{Kesimpulan}
Berdasarkan hasil dari eksperimen, maka dapat ditarik kesimpulan sebagai berikut:
\begin{enumerate}
	\item Kontribusi peneliti terdapat pada penemuan dibutuhkannya tahap interpolasi \emph{preprocessing} sebelum \emph{active contour} dijalankan.
	\item Hasil dari deteksi keliling luka menggunakan \emph{snake} versi \emph{integer} yang kurva akhirnya berhasil menutupi luka berjumlah 12 data (dari 71 data) sedangkan untuk yang versi interpolasi berjumlah 44 data (dari 71 data). Hal ini menunjukkan bahwa data yang berhasil dideteksi menggunakan \emph{snake} interpolasi lebih banyak dibandingkan versi \emph{integer}.
	\item Hasil dari deteksi keliling luka menggunakan \emph{snake} versi \emph{integer} untuk semua kategori menghasilkan 12 data (dari 71 data) yang kurva akhirnya berhasil menutupi luka dengan nilai akurasi rata-rata 77.18\%.
	\item Hasil dari deteksi keliling luka menggunakan \emph{snake} versi \emph{integer} untuk semua kategori menghasilkan 44 data (dari 71 data) yang kurva akhirnya berhasil menutupi luka dengan nilai akurasi rata-rata 86.1\%.
\end{enumerate}

\section{Saran}
Pemilihan metode segmentasi citra menggunakan algoritma \emph{Grabcut} membuka peluang 
untuk melanjutkan penelitian dengan tujuan meningkatkan kinerja \emph{Grabcut}. 
Salah satu pendekatan yang mungkin adalah mengganti atau memodifikasi metode 
segmentasi dengan metode citra alternatif, dengan harapan dapat meningkatkan 
akurasi hasil yang diperoleh dari penggunaan algoritma \emph{Grabcut}.

% Baris ini digunakan untuk membantu dalam melakukan sitasi
% Karena diapit dengan comment, maka baris ini akan diabaikan
% oleh compiler LaTeX.
\begin{comment}
\bibliography{daftar-pustaka}
\end{comment}