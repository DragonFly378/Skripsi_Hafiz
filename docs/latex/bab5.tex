%!TEX root = ./template-skripsi.tex
%-------------------------------------------------------------------------------
%                            	BAB IV
%               		KESIMPULAN DAN SARAN
%-------------------------------------------------------------------------------

\chapter{KESIMPULAN DAN SARAN}

\section{Kesimpulan}
Berdasarkan percobaan yang telah dilakukan, maka didapatkanlah hasil kesimpulan sebagai berikut:
\begin{enumerate}
	% \item Kontribusi peneliti terdapat pada penggunaan metode Grabcut pada 
	% segmentasi 	citra luka serta membandingkan akurasinya dengan metode sebelumnya 
	% yang dilakukan \cite{Rizki:2022}.

	\item Kontribusi peneliti terdapat pada penggunaan metode Grabcut pada 
	segmentasi 	citra luka serta mendapatkan nilai akurasinya.

	% \item Hasil dari deteksi keliling luka menggunakan \emph{Grabcut} berhasil 
	% menutupi luka berjumlah 70 data (dari 71 data) sedangkan metode \emph{snake} 
	% versi \emph{integer} yang kurva akhirnya berhasil menutupi luka berjumlah 12 
	% data (dari 71 data) dan untuk yang versi interpolasi berjumlah 44 data (dari 71 data). 
	% Hal ini menunjukkan bahwa data yang berhasil dideteksi menggunakan \emph{Grabcut} 
	% lebih banyak dibandingkan menggunakan \emph{snake}.

	\item Hasil dari deteksi keliling luka menggunakan \emph{Grabcut} untuk semua kategori 
	berhasil menutupi luka berjumlah 70 data (dari 71 data).
	
	\item Hasil dari deteksi keliling luka menggunakan \emph{Grabcut} untuk semua 
	kategori menghasilkan 70 data (dari 71 data) yang kurva akhirnya berhasil 
	menutupi luka dengan nilai akurasi rata-rata 89,36\%.

\end{enumerate}

\section{Saran}
Pemilihan metode segmentasi citra menggunakan algoritma \emph{Grabcut} membuka peluang 
untuk melanjutkan penelitian dengan tujuan meningkatkan kinerja \emph{Grabcut}. 
Salah satu pendekatan yang mungkin adalah mengganti atau memodifikasi metode 
segmentasi dengan metode citra alternatif, dengan harapan dapat meningkatkan 
akurasi hasil yang diperoleh dari penggunaan algoritma \emph{Grabcut}.

% Baris ini digunakan untuk membantu dalam melakukan sitasi
% Karena diapit dengan comment, maka baris ini akan diabaikan
% oleh compiler LaTeX.
\begin{comment}
\bibliography{daftar-pustaka}
\end{comment}