%!TEX root = ./template-skripsi.tex
%-------------------------------------------------------------------------------
% 								BAB I
% 							LATAR BELAKANG
%-------------------------------------------------------------------------------

\chapter{PENDAHULUAN}

\section{Latar Belakang Masalah}

% Intro
Luka merupakan kerusakan atau gangguan  yang terjadi pada stuktur anatomi kulit. 
Hal ini sering kita temui pada permukaan kulit atau pada integritas epitel kulit,
luka berkisar dari kerusakan yang bervariasi mulai dari kerusakan sederhana atau
terjadi lebih dalam, bahkan bisa saja meluas ke jaringan subkutan yang berdampak 
pada struktur lain seperti tendon, pembuluh darah, otot, saraf, organ parenkim, dan 
hingga sampai tulang. Luka timbul karena adanya proses patologis secara internal 
maupun eksternal, apapun penyebab dan bentuknya, luka dapat merusak jaringan dan 
mengganggu sistem yang ada didalamnya, respon yang ditimbulkan oleh luka secara 
fisiologis di antaranya menyebabkan pendarahan, kontraksi pembuluh darah dengan 
koagulasi, aktivasi komplemen serta respon inflamasi (\cite{Velnar:2009}).


% Klasifikasi luka
Luka dapat diklasifikasikan dalam berbagai kriteria. Berdasarkan waktu penyembuhan 
luka dibagi menjadi dua, yaitu luka akut dan luka kronis. ~\cite{Velnar:2009} dalam 
penelitiannya mengatakan luka akut adalah luka yang dapat penyembuhan secara mandiri 
dan berlangsung secara normal dengan proses penyembuhan yang membutuhkan waktu yang 
teratur, luka akut berakhir dengan hasil dari restorasi fungsi anatomis. 
Luka kronis ialah luka yang proses penyembuhannya gagal berjalan dengan normal, 
proses penyembuhan luka kronis tidak dapat diperbaiki dengan cepat dan teratur, 
hal ini dikarenakan adanya gangguan oleh beberapa faktor dalam tahap fase hemositasis, 
peradangan, proliferasi atau \emph{remodelling}. Luka kronis dapat disebabkan oleh berbagai 
penyebab, di antaranya naturopati, tekanan, insufisiensi arteri dan vena, diabetes melitus, 
luka bakar dan vaskulitis.


% Penyembuhan luka
Proses penyembuhan luka adalah proses kompleks dengan serangkaian interaksi beragan 
antara sistem imunologi dan biologis, penyembuhan luka ini terdiri dari berbagai 
fase dengan langkah dan peristiwa yang berlangsung dengan perlahan dan dilakukan 
secara sistematis sehingga muncul berbagai jenis sel badi dasar luka selama proses 
penyembuhan berlangsung (\cite{Velnar:2009}). Ketika seseorang mengalami luka 
dibagian jaringan kulit, penanganan yang biasanya dilakukan adalah menutup luka 
tersebut agar tidak terjadi pendarahan terus menerus, hal tersebut bisa dilakukan 
jika luka yang dialami ialah luka kecil dan hanya dipermukaan saja, namun jika luka 
tersebut merupakan luka kronis, maka disarankan untuk melakukan penanganan dan pengkajian 
ke rumah sakit agar segera dilakukan tindakan oleh dokter atau perawat.  


% luka kronis
Luka kronis menjadi salah satu permasalahan bagi beberapa pihak. Bagi pasien yang 
memiliki luka kronis khususnya akibat penyakit Diabetes Melitus (DM) akan menghabiskan 
banyak biaya dalam pengobatannya. \cite{Wang:2015} dalam penelitiannya menyebutkan 
di Amerika Serikat jutaan pasien penderita luka kronis mengeluarkan miliaran uang 
tiap tahunnya,  pasien penderita luka kronis akibat Diabetes Melitus (DM) saja bisa 
menghabiskan 38 miliar dolar, kebanyakan di antaranya biaya rawat inap dan operasi di 
rumah sakit, dan biaya perawatan jangka panjang yaitu perawatan dari rumah secara 
berkala. Di Amerika tercatat jumlah pasien penderita diabetes mencapai 20 juta dan 
diperkirakan pada tahun 2030 jumlahnya akan naik dua kali lipat (\cite{Han:2017}). 
Tentu saja dengan jumlah begitu banyak maka perawat dan rumah sakit yang menangani 
perawatan luka akan memakan banyak waktu  dan mengeluarkan banyak biaya (\cite{Wang:2015}), 
hal ini mengakibatkan tenaga perawat yang menangani pasien penderita luka kronis 
membutuhkan banyak waktu. 


% Tahapan penyembuhan
Proses penyembuhan luka terjadi dalam beberapa tahapan, seorang perawat luka wajib 
memberikan asesmen perawatan luka sesuai prosedur medis agar keadaan luka segera 
membaik dan menghindari terjadnya infeksi. Proses penyembuhan yang dilakukan pertama 
kali ialah membersihkan dan dibalut dengan benar, setelah luka dibersihkan maka 
akan dilakukan metode debridemen luka, hal ini bertujuan untuk mengangkat jaringan 
(nektrotik) yang mati, terinfeksi dan penebalan pada jaringan kulit (hiperkeratoris), 
membentuk dasar penyembuhan luka. Debridemen memiliki fungsi yang penting dalam 
asesmen luka karena akan mempercepat proses penyembuhan luka, debridemen yang ada 
pada luka kronis berfungsi sebagai penanganan kelainan medis dan mengubah kronis 
tersebut menjadi luka akut, setelah menjadi luka akut maka penyembuhan akan kembali 
normal (\cite{Velnar:2009}).


% Asesmen luka
Proses asesmen luka dilakukan bertahap, tindakan yang diberikan tiap tahap penyembuhan 
dilakukan berdasarkan kajian evaluasi luka, kajian ini memantau proses penyembuhan 
luka secara berkala (\cite{Silva:2021}). Metode evaluasi luka biasanya menggunakan 
metode invasif (kontak) dan non-invasif (non-kontak) (\cite{ManoharDhane:2017}).

Hal penting yang dijadikan sebagai indikator untuk penyembuhan luka ialah dengan 
memperhatikan ukuran luka, di antaranya perubahan luas, kedalaman dan jenis jaringan 
yang luka (\cite{Silva:2021}). Seorang perawat luka melakukan inspeksi kontak 
luka dengan mengukur luka dilakukan secara manual yaitu dengan bantuan penggaris 
luka dan label perekat yang bersentuhan langsung dengan luka, dalam penelitiannya 
\cite{Silva:2021} menyebutkan teknik ini cenderung dapat menyebabkan resiko infeksi, 
mengganggu kenyamanan dan memperburuk kondisi klinis pasien, ditambah penggunaan 
teknik-teknik tersebut tidak akurat, tidak konsisten, dan tentu mengalami kekurangan 
standar penilaian. Standar pengukuran dengan metode manual tersebut memiliki tingkat 
kesalahan yang cukup tinggi yaitu sekitar 44 persen  (\cite{Rizki:2022}).


Para ilmuwan telah banyak melakukan penelitian mengenai hal ini, agar proses asesmen 
dan kajian luka dapat dilakukan secara efektif dan efisien mulai dari waktu, tenaga, 
hingga pengeluaran biaya yang cukup banyak, teknologi dibidang pemrosesan citra gambar 
menjadi hal yang memungkinkan saat ini untuk dikembangkan, dengan menggunakan pemrosesan 
gambar yang dibantu oleh pembelajaran mesin memungkinkan melakukan analisis gambar 
luka oleh program komputer (\cite{Wang:2015}). 


\cite{Silva:2021} berpendapat bahwa pengembangan citra digital untuk melakukan 
evaluasi luka merupakan alternatif yang sangat penting, teknik ini akan menginformasikan 
analisis yang lebih objektif dan reliabel. Mereka melakukan penelitian mengenai 
pengusulan untuk menentukan area luka dengan menggunakan pengklasifikasi berbasis 
citra medis yaitu  \emph{Support Vector Mechine} (SVM) serta mengkombinasikannya dengan 
metode \emph{GrabCut} untuk segmentasi area yang terkena luka. Metode segmentasi gambar 
ini sepenuhnya otomatis serta perawat tidak perlu melakukan kontak langsung dengan 
objek, tingkat akurasinya diperkirakan mencapai 96 persen, sensitifitas sebesar 94 persen, 
spesifisitas 97 persen, tingkat presisi 94 persen dan interaksi penyatuan 89 persen.


Langkah pertama seorang perawat dalam melakukan asesmen luka digital ialah dengan 
mengambil foto luka menggunakan kamera telepon pintar atau tablet nya, ketika ada 
dua foto dengan pose yang sama namun dengan perangkat kamera yang berbeda, warna 
yang dihasilkan kemungkinan akan berbeda. Untuk menangani hal tersebut adalah setiap 
kamera menggunakan \emph{device independent} sRGB. Zaman sekarang sudah banyak vendor kamera 
telepon pintar menyediakan mode ini namun dengan pengaturan pewarnaan RGB mereka 
masing-masing. Oleh karena ini dibutuhkan adanya kalibrasi dengan melakukan 
transformasi citra menjadi ruang warna CIE, kemudian menjadi sRGB dengan serangkaian 
optimalisasi (\cite{Rizki:2022}).

\cite{Silva:2021} menjelaskan dalam karyanya bahwa parameter yang digunakan untuk 
segmentasi daerah yang terkena luka ialah perbedaan warna antara kulit dan luka, 
berdasarkan ambang batas semi otomatis dalam ruang pewarnaan RGB, 
\cite{ManoharDhane:2017} menggunakan metode pengelompokan spektral fuzzy untuk segmentasi warna 
berdasarkan tingkat kesamaan fuzzy (tingkat abu-abu) yang dihitung di atas gambar, 
hasil menunjukkan pada 70 gambar mencapai akurasi 92 persen, sensitifitas 87 persen, dan spesifisitas 96 persen.


Sebenarnya ada beberapa metode yang bisa digunakan untuk melakukan image processing 
dengan bantuan \emph{Support Vector Mechine} (SVM) seperti kesamaan fuzzy, \emph{active contour} 
(snake), dan \emph{region-of-interest} (ROI), dan \emph{GrabCut}. Metode-metode yang pernah dilakukan 
merupakan pendekatan untuk menghasilkan objek luka yang akurat \cite{Garcia-Zapirain:2017}. 
Beberapa di antara metode yang disebutkan juga memiliki keterbatasan yang beragam, 
mulai dari inisiasi awal gambar yang manual, atau intensitas warna yang bergantung 
dari intensitas gradien. 


\cite{Rizki:2022} dalam penelitiannya yaitu pendeteksi luka menggunakan metode 
\emph{active contour} (snake) dan \emph{active contour} (snake) dengan ditambah 
interpolasi, dalam penelitiannya mencari  \emph{ground truth} dalam objek-objek gambar 
luka pasien, ketika dilakukan pemrosesan gambar dengan metode yang dijalankan, \emph{ground truth} menunjukkan hasil yang kurang maksimal dimana hasil dari deteksi 
luka dengan metode snake versi integer hanya berhasil menutupi luka berjumlah 
12 data dari total 71 data yang tersedia, sedangkan metode snake interpolasi 
berjumlah 44 data dari total 71 data yang tersedia. \cite{Silva:2021} melakukan 
penelitian mengenai pengolahan citra gambar dengan menggunakan \emph{Support Vector Mechine} 
(SVM) dengan menggunakan \emph{GrabCut}, dalam tulisannya menerangkan bagaimana cara segmentasi 
gambar luka dengan menggunakan \emph{GrabCut},  setidaknya ada lima tahapan dalam prosesnya, 
di antaranya segmentasi superpiksel, ekstraksi fitur, persiapan data, klasifikasi 
dan terakhir yaitu segmentasi luka. 


Segmentasi superpiksel merupakan teknik untuk merepresentasi ringkas dari gambar 
menjadi kelompok piksel yang lebih kecil sesuai dengan spasial dan kriteria warna. 
Segmentasi superpiksel  ini digunakan oleh beberapa metode di antaranya SEED, LSC 
dan SLIC. Metode ini menghasilkan beberapa informasi dari gambar luka seperti tepi 
luka dan parameter yang digunakan untuk metode selanjutnya (\cite{Silva:2021}). 
\cite{Wang:2015} menggunakan gambar berukuran 320 x 640 piksel, hal ini dikarenakan 
untuk memangkas biaya komputasi dalam pemrosesan, sampel diambil secara acak, sementara 
SVM linier untuk melatih data yang tersedia. Setelah data telah tersegmentasi menjadi 
superpiksel, maka dilakukan proses klasifikasi dengan menggunakan \emph{Support Vector 
Mechine} (SVM), hal ini diperlukan karena pada ditahap ini banyak terdapat superpiksel 
berada di sekitar luka pada kulit dan dikhawatirkan dapat salah dalam klasifikasi 
sehingga menurunkan tingkat akurasi segmentasi. Proses selanjutnya dalam pengolahan 
citra gambar luka ialah segmentasi luka, salah satu metode yang paling populer untuk 
segmentasi citra gambar ialah menggunakan algoritma \emph{GrabCut}, teknik ini merupakan 
teknik yang bekerja berdasarkan analisis statistik serta teori grafik, hal ini 
bertujuan untuk memisahkan suatu objek inti gambar dari objek sisa di sekitar gambar, 
pada akhirnya data telah menghasilkan suatu objek citra yang berisi area luka beserta 
komponen yang ada pada data tersebut. (\cite{Nugraha:2022}) menggunakan metode 
\emph{GrabCut} dalam penelitiannya mengenai ekstraksi latar depan citra ikan, hasil yang 
didapatkan dari dataset untuk dilakukan uji coba bahwa apabila \emph{GrabCut} diuji pada 
data citra multi objek, maka objek yang ada pada data tersebut akan gagal diseleksi 
oleh \emph{GrabCut}, namun jika data tersebut hanya terdapat satu objek maka \emph{GrabCut} berhasil 
melakukan seleksi citra dengan baik dan menghasilkan data yang bagus.


Di dalam penelitian ini, peneliti akan melakukan penerapan algoritma \emph{GrabCut} pada 
pemrosesan citra gambar yang akan menghasilkan segmentasi area luka, metode yang 
peneliti pilih berdasarkan hasil dari penelitian Muhammad Rizki dimana \emph{ground truth} (area sebenarnya) 
yang dihasilkan lebih banyak yang gagal dibandingkan yang berhasil, sehingga peneliti 
tertarik untuk mengganti metode yang dijalankan untuk segmentasi area keliling luka 
kronis. Selain itu peneliti memilih metode \emph{GrabCut} untuk dijadikan sebagai 
penelitian dikarenakan metode ini sangat cocok untuk citra gambar luka dimana yang 
merupakan citra satu objek \emph{(single object)}. Tahap pertama peneliti akan menandai 
daerah yang mencakup objek luka, kemudian objek akan dilakukan pengujian metode 
\emph{GrabCut} pada citra tunggal (\emph{single object}). Dalam penelitian ini 
peneliti akan menggunakan dataset citra yang tersedia di repositori \emph{https://github.com/mekas/InjuryDetection.}
Dataset citra ini berasal dari penelitian luka Ns. Ratna Aryani M.Kep, tahun 2018 
(\cite{Aryani:2018}). Diharapkan dalam penelitian ini mendapatkan hasil berupa nilai 
akurasi antara metode \emph{GrabCut} dengan hasil citra referensi.

\section{Rumusan Masalah}
Berdasarkan uraian pada latar belakang yang diutarakan di atas, maka perumusan 
masalah pada penelitian ini adalah Bagaimana cara mendeteksi keliling luka dengan 
menggunakan metode \emph{GrabCut}?

\section{Pembatasan Masalah}
Adapun beberapa pembatasan telah diterapkan dalam penelitian ini guna menjaga fokus 
pada permasalahan yang dijelaskan sebelumnya. Berikut adalah batasan-batasan yang 
telah diterapkan:

\begin{enumerate}
	\item Mendeteksi area keliling luka kronis menggunakan metode \emph{GrabCut} dengan data 
	citra luka yang didapat dari penelitian luka Ns. Ratna Aryani, M.Kep, tahun 2018. 
	\item Penelitian ini hanya berfokus pada pemisahan antara objek utama dan 
	latar belakang pada citra gambar luka.
	\item Objek utama, yaitu area luka, diharapkan hanya terdapat satu area luka pada setiap citra.
	\item Citra-citra yang digunakan dalam penelitian ini adalah citra digital yang menggunakan sistem warna RGB.
	\item Ukuran panjang citra yang dijadikan objek penelitian adalah 320 piksel.
	\item Bahasa pemrograman yang digunakan dalam penelitian ini adalah Python v 3.9
\end{enumerate}

\section{Tujuan Penelitian}
Penelitian ini bertujuan untuk membuat sistem untuk mendeteksi objek luka pada citra luka
dengan menggunakan metode \emph{GrabCut}.

\section{Manfaat Penelitian}
\begin{enumerate}
	\item Bagi peneliti
	\begin{itemize}
		\item Meningkatkan wawasan dan pengalaman praktis terkait deteksi latar 
		depan pada citra luka melalui penggunaan metode \emph{GrabCut}.

		\item Untuk memenuhi persyaratan kelulusan dalam program Sarjana (S1) di 
		Program Studi Ilmu Komputer, Fakultas Matematika dan Ilmu Pengetahuan Alam, 
		Universitas Negeri Jakarta
	\end{itemize}
		
	\item Bagi Instansi Terkait
	
	Metode yang diajukan diharapkan dapat membuka peluang untuk diajukan ke instansi 
	kesehatan terkait dalam proses pengkajian luka kronis.

	\item Bagi Ilmu Pengetahuan

	\begin{itemize}
		\item Mahasiswa
		
		peneliti berharap penelitian ini dapat digunakan sebagai sumber penunjang 
		referensi, khususnya Pustaka tentang deteksi keliling luka kronis dengan 
		menggunakan metode \emph{GrabCut}.

		\item Bagi Peneliti Selanjutnya
	
		Diharapkan penelitian ini dapat digunakan sebagai dasar atau kajian awal 
		bagi peneliti lain yang ingin meneliti permasalahan yang sama.
	\end{itemize}

	\item Bagi Universitas Negeri Jakarta
	
	Menjadi pertimbangan dan evaluasi akademik khususnya Program Studi Ilmu Komputer 
	dalam penyusunan skripsi sehingga dapat meningkatkan kualitas akademik di program 
	studi Ilmu Komputer Universitas Negeri Jakarta serta meningkatkan kualitas lulusannya.
			
\end{enumerate}

% Baris ini digunakan untuk membantu dalam melakukan sitasi
% Karena diapit dengan comment, maka baris ini akan diabaikan
% oleh compiler LaTeX.
\begin{comment}
\bibliography{daftar-pustaka}
\end{comment}