\chapter*{\centering{\large{ABSTRAK}}}
\singlespacing{}

\textbf{MUHAMMAD HAFIZ HISBULLAH}, Deteksi Area Keliling Luka Kronis Dengan 
Menggunakan Algoritma \emph{GrabCut}. Skripsi, Program Studi Ilmu Komputer, Fakultas Matematika dan Ilmu Pengetahuan Alam, Universitas Negeri Jakarta. Januari 2024.
\\
\\
Luka kronis merupakan masalah kesehatan yang kompleks, khususnya bagi pasien dengan 
penyakit seperti Diabetes Melitus (DM). Proses penyembuhan luka melibatkan asesmen 
yang cermat dan pengelolaan yang efektif, namun metode manual dalam pengukuran 
luka seringkali tidak akurat dan memakan waktu. Skripsi ini bertujuan untuk penggunaan 
metode \emph{GrabCut} dalam pemrosesan citra untuk mendeteksi area keliling 
luka kronis. Metode ini memiliki potensi untuk memberikan analisis yang objektif 
dan reliabel dalam asesmen luka. Dengan memanfaatkan teknologi pemrosesan gambar 
dan pembelajaran mesin, kami mencoba mengatasi keterbatasan metode manual dengan 
menggunakan algoritma \emph{GrabCut}. Langkah pertama melibatkan pengambilan foto 
luka menggunakan perangkat seluler, diikuti oleh kalibrasi warna untuk memastikan 
konsistensi. Metode \emph{GrabCut} kemudian diterapkan untuk mengsegmentasi area 
luka, dengan fokus pada pemisahan antara objek utama (luka) dan latar belakang. 
Hasil akhir menunjukkan bahwa data yang berhasil disegmentasi menggunakan \emph{Grabcut}
sebanyak 70 (dari 71 data) serta menghasilkan rata-rata nilai akurasi \emph{Grabcut} 
89.36\% 
\\
\\
\textbf{Kata kunci:} luka kronis, \emph{GrabCut}, pemrosesan citra, asesmen luka, pengelolaan luka.
