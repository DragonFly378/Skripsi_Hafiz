\chapter*{\centering{\large{KATA PENGANTAR}}}
\onehalfspacing{}
Dengan penuh syukur, peneliti ingin mengungkapkan rasa terima kasih kepada Allah SWT, 
atas berkat-Nya yang memungkinkan peneliti menyelesaikan proposal skripsi berjudul 
\textit{DETEKSI AREA KELILING LUKA KRONIS DENGAN MENGGUNAKAN ALGORITMA GRABCUT}.

Semoga penghargaan ini mencerminkan segenap rasa syukur peneliti atas bimbingan, 
dorongan, dan inspirasi yang telah diberikan. Harapannya, tugas akhir ini dapat 
memberikan manfaat yang nyata bagi semua pihak yang terlibat, dan khususnya, dapat 
menjadi langkah maju dalam peningkatan pengetahuan dan pemahaman dalam bidang yang 
diteliti. 

Peneliti juga berdoa semoga Allah SWT senantiasa membalas kebaikan dan 
keikhlasan dari semua pihak yang telah turut serta membantu hingga proposal skripsi
ini dapat selesai, oleh karena itu dalam kesempatan ini, dengan kerendahan hati 
penulis mengucapkan banyak terima kasih kepada:

\begin{enumerate}

	\item{Yth. Para petinggi di lingkungan FMIPA Universitas Negeri Jakarta.}
	\item{Yth. Ibu Dr. Ria Arafiyah, M.Si selaku Koordinator Program Studi Ilmu
		Komputer.}
	\item{Yth. Bapak Muhammad Eka Suryana, M.Kom selaku Dosen Pembimbing I yang
		telah membimbing, mengarahkan, serta memberikan saran dan koreksi terhadap
		proposal skripsi ini.}
	\item{Yth. Bapak Drs. Mulyono, M.Kom selaku Dosen Pembimbing II yang telah
		membimbing, mengarahkan, serta memberikan saran dan koreksi terhadap
		proposal skripsi ini.}
	\item{Kedua orang tua dan kakak peneliti yang telah mendukung dan memberikan 
		semangat serta doa untuk peneliti.}
	\item{Teman-teman Warung Tegang yang walau dengan segala keanehannya selalu
		memberikan semangat dan motivasi bagi peneliti.}
	\item{Jasinga yang menjadi tempat perjuangan peneliti dalam menyelesaikan skripsi 
		sekaligus tempat \emph{refreshing} peneliti karena pemandangannya yang indah.}
	\item{Teman-teman Program Studi Ilmu Komputer 2019 yang telah memberikan 
		dukungan dan memiliki andil dalam penelitian proposal skripsi ini.}
	
\end{enumerate}

Dengan penuh kesederhanaan, peneliti mengakui bahwa dalam penyusunan proposal skripsi 
ini, belum mencapai kesempurnaan karena terbatasnya pengetahuan dan pengalaman 
yang dimiliki. Karena itulah, peneliti dengan hati yang lapang menerima kritik 
serta saran yang konstruktif, yang datang bagai bunga-bunga pengharapan.

Sebagai perjalanan ini mencapai akhir, peneliti mengharapkan tugas akhir ini akan 
menjadi ladang berkah bagi semua pihak yang terlibat, tak terkecuali diri peneliti 
sendiri. Dalam doa yang dipanjatkan, semoga Allah SWT selalu melipatgandakan kebaikan 
bagi semua yang telah berperan serta membantu peneliti meniti jalan menuju puncak 
kesuksesan dalam menyelesaikan proposal skripsi ini.

\vspace{2cm}

\begin{tabular}{p{7.5cm}c}
	&Jakarta, 11 Januari 2024\\
	&\\
	&\\
	&\\
	&Mochammad Hafiz Hisbullah
\end{tabular}
